latex
\documentclass{article}
\usepackage{amsmath}
\usepackage{amsfonts}
\usepackage{amssymb}
\usepackage{geometry}
\geometry{a4paper, margin=1in}

\title{Plant Reproduction: A Question Paper}
\author{Your Name}
\date{\today}

\begin{document}

\maketitle

\section*{Section 1: Multiple-Choice Questions (Questions 1 to 6)}

\begin{enumerate}
    \item What is the outer layer of the pollen grain called?
        \begin{itemize}
            \item A. Intine
            \item B. Exine
            \item C. Cellulose
            \item D. Pectin
        \end{itemize}
    \item What substance makes the exine resistant to harsh conditions?
        \begin{itemize}
            \item A. Cellulose
            \item B. Pectin
            \item C. Sporopollenin
            \item D. Lignin
        \end{itemize}
    \item What is the function of the germ pore in a pollen grain?
        \begin{itemize}
            \item A. Nutrient storage
            \item B. Pollen tube formation
            \item C. Protection from UV radiation
            \item D. Water absorption
        \end{itemize}
    \item  Which cells are responsible for providing nourishment to the generative cell in a mature pollen grain?
        \begin{itemize}
            \item A. Generative cells
            \item B. Vegetative cells
            \item C. Synergid cells
            \item D. Antipodal cells
        \end{itemize}
    \item In how many percent of angiosperms are pollen grains shed at the 2-celled stage?
        \begin{itemize}
            \item A. 20%
            \item B. 40%
            \item C. 60%
            \item D. 80%
        \end{itemize}
    \item  What is the name of the grass that is a common cause of pollen allergies in India?
        \begin{itemize}
            \item A. Wheat
            \item B. Rice
            \item C. *Parthenium* (Carrot grass)
            \item D. *Solanum*
        \end{itemize}
\end{enumerate}


\section*{Section 2: Assertion-Reason Questions (Questions 7 and 8)}

\begin{enumerate}
    \item \textbf{Assertion (A):}\ The exine of the pollen grain is a hard layer. \\
    \textbf{Reason (R):}\ The exine is composed of sporopollenin, a highly resistant organic material.
        \begin{itemize}
            \item A. Both A and R are true, and R is the correct explanation of A.
            \item B. Both A and R are true, but R is not the correct explanation of A.
            \item C. A is true, but R is false.
            \item D. A is false, but R is true.
            \item E. Both A and R are false.
        \end{itemize}
    \item \textbf{Assertion (A):}\ Pollen grains can be stored for long periods in liquid nitrogen. \\
    \textbf{Reason (R):}\  Liquid nitrogen maintains low temperatures, preventing the degradation of pollen viability.
        \begin{itemize}
            \item A. Both A and R are true, and R is the correct explanation of A.
            \item B. Both A and R are true, but R is not the correct explanation of A.
            \item C. A is true, but R is false.
            \item D. A is false, but R is true.
            \item E. Both A and R are false.
        \end{itemize}
\end{enumerate}


\section*{Section 3: Short-Answer Questions (Questions 9 and 10)}

\begin{enumerate}
    \item Briefly describe the structure of a mature pollen grain, including the vegetative and generative cells.
    \item Explain the significance of pollen banks in crop improvement programs.
\end{enumerate}

\section*{Answer Key}
\begin{enumerate}
    \item B
    \item C
    \item B
    \item B
    \item C
    \item C
    \item A
    \item A
    \item \textit{(Short answer -  Describe the two-celled structure, vegetative cell's role in nutrition and pollen tube growth, generative cell's role in male gamete formation.)}
    \item \textit{(Short answer - Explain the preservation of pollen for future breeding programs, similar to seed banks.)}
\end{enumerate}

\end{document}